%% LyX 1.3 created this file.  For more info, see http://www.lyx.org/.
%% Do not edit unless you really know what you are doing.
\documentclass[english, 12pt]{article}
\usepackage{times}
%\usepackage{algorithm2e}
\usepackage{url}
\usepackage{bbm}
\usepackage[T1]{fontenc}
\usepackage[latin1]{inputenc}
\usepackage{geometry}
\geometry{verbose,letterpaper,tmargin=2cm,bmargin=2cm,lmargin=1.5cm,rmargin=1.5cm}
\usepackage{rotating}
\usepackage{color}
\usepackage{graphicx}
\usepackage{amsmath, amsthm, amssymb}
\usepackage{setspace}
\usepackage{lineno}
\usepackage{hyperref}
\usepackage{bbm}
\usepackage{makecell}
\usepackage{placeins}
\usepackage{subcaption}

%\renewcommand{\arraystretch}{1.8}

%\linenumbers
%\doublespacing
\onehalfspacing
\usepackage[numbers,super]{natbib}

%Pour les rajouts
\usepackage{color}
\definecolor{trustcolor}{rgb}{0,0,1}

\usepackage{dsfont}
\usepackage[warn]{textcomp}
\usepackage{adjustbox}
\usepackage{multirow}
\usepackage{subcaption}
\usepackage{graphicx}
\graphicspath{{../figures/}}
\DeclareMathOperator*{\argmin}{\arg\!\min}

\let\tabbeg\tabular
\let\tabend\endtabular
\renewenvironment{tabular}{\begin{adjustbox}{max width=0.95\textwidth}\tabbeg}{\tabend\end{adjustbox}}

\makeatletter

%%%%%%%%%%%%%%%%%%%%%%%%%%%%%% LyX specific LaTeX commands.
%% Bold symbol macro for standard LaTeX users
%\newcommand{\boldsymbol}[1]{\mbox{\boldmath $#1$}}

%% Because html converters don't know tabularnewline
\providecommand{\tabularnewline}{\\}
\renewcommand*{\arraystretch}{1.2}

\usepackage{babel}
\makeatother


\begin{document}

\renewcommand{\thefigure}{S\arabic{figure}}
\setcounter{figure}{0}
\renewcommand{\thetable}{S\arabic{table}}
\setcounter{table}{0}
\renewcommand{\theequation}{S\arabic{equation}}
\setcounter{equation}{0}

%%%%%%%%%%%%%%%%%%%%%%%%%%%%%%%%%%%%%%%%%%%%%%%%%%%%%%%%%%%%%%%%%%%%%%%%%%%%%%%%

\begin{figure}[p]
\centerline{\includegraphics[width=\textwidth]{est_h2_one}}
\caption{Inferred SNP heritability $h^2$ in simulations with continuous outcomes. Horizontal lines represent the true simulated values. The 95\% confidence interval for the LDpred2-auto estimate is obtained from the 2.5\% and 97.5\% quantiles of all the $h^2$ estimates from the iterations (after burn-in) of the chains kept. The 95\% confidence interval for the LD Score regression estimate is obtained from $\pm$1.96 times its standard error. \label{fig:simu_h2}}
\end{figure}

\begin{figure}[p]
\centerline{\includegraphics[width=\textwidth]{est_p_one}}
\caption{Inferred polygenicity $p$ in simulations with continuous outcomes. \label{fig:simu_p}}
\end{figure}

\begin{figure}[p]
\centerline{\includegraphics[width=\textwidth]{est_alpha_one}}
\caption{Inferred $\alpha$ in simulations with continuous outcomes. \label{fig:simu_alpha}}
\end{figure}

\begin{figure}[p]
\centerline{\includegraphics[width=\textwidth]{postp_calib}}
\caption{Calibration of per-variant posterior probabilities of being causal in simulations with continuous outcomes. \label{fig:simu_postp_calib}}
\end{figure}

\begin{figure}[p]
\centerline{\includegraphics[width=\textwidth]{local_h2_calib}}
\caption{Calibration of per-block heritability estimates in simulations with continuous outcomes. \label{fig:simu_h2_calib}}
\end{figure}

\begin{figure}[p]
\centerline{\includegraphics[width=\textwidth]{est_r2_one}}
\caption{Inferred predictive performance $r^2$ in simulations with continuous outcomes. \label{fig:simu_r2}}
\end{figure}


%%%%%%%%%%%%%%%%%%%%%%%%%%%%%%%%%%%%%%%%%%%%%%%%%%%%%%%%%%%%%%%%%%%%%%%%%%%%%%%%

\FloatBarrier

[ ADD THE SAME, BUT AVERAGING OVER 10 SIMUS -> CI OF THE MEAN ]

\FloatBarrier

%%%%%%%%%%%%%%%%%%%%%%%%%%%%%%%%%%%%%%%%%%%%%%%%%%%%%%%%%%%%%%%%%%%%%%%%%%%%%%%%

\begin{figure}[p]
	\centerline{\includegraphics[width=\textwidth]{est_binary_h2_one}}
	\caption{All h2 estimates were transformed to the liability scale with K\_pop=K and K\_GWAS either K or 0.5 (when using Neff) \label{}}
\end{figure}

\begin{figure}[p]
	\centerline{\includegraphics[width=\textwidth]{est_binary_p_one}}
	\caption{ \label{}}
\end{figure}

\begin{figure}[p]
	\centerline{\includegraphics[width=\textwidth]{est_binary_alpha_one}}
	\caption{ \label{}}
\end{figure}

\begin{figure}[p]
	\centerline{\includegraphics[width=\textwidth]{est_binary_r2_one}}
	\caption{Transformed both r2 to the liability scale, where the inferred one using Neff used K\_GWAS=0.5  \label{}}
\end{figure}

\begin{figure}[p]
	\centerline{\includegraphics[width=\textwidth]{binary_postp_calib}}
	\caption{ \label{}}
\end{figure}

%%%%%%%%%%%%%%%%%%%%%%%%%%%%%%%%%%%%%%%%%%%%%%%%%%%%%%%%%%%%%%%%%%%%%%%%%%%%%%%%

\FloatBarrier

\begin{figure}[p]
	\centerline{\includegraphics[width=0.9\textwidth]{ukbb_h2}}
	\caption{ \label{}}
\end{figure}

\begin{figure}[p]
	\centerline{\includegraphics[width=\textwidth]{ukbb_h2_p}}
	\caption{ \label{}}
\end{figure}

\begin{figure}[p]
	\centerline{\includegraphics[width=\textwidth]{ukbb_h2_alpha}}
	\caption{ \label{}}
\end{figure}

\begin{figure}[p]
	\centerline{\includegraphics[width=\textwidth]{ukbb_r2}}
	\caption{Height has estimated h2 of 0.555 and 0.626 -> probably too large, and why estimated r2 is too large as well. The other outliers is 695.4 (Lupus), with a very small number of cases -> should probably stop using 16 PCs \label{}}
\end{figure}

\begin{figure}[p]
\centerline{\includegraphics[width=\textwidth]{ukbb_compare_hm3_small}}
\caption{ \label{}}
\end{figure}

\begin{figure}[p]
\centerline{\includegraphics[width=\textwidth]{ukbb_compare_hm3_altpop}}
\caption{ \label{}}
\end{figure}

\begin{figure}[p]
	\centerline{\includegraphics[width=\textwidth]{ukbb_compare_hm3_plus}}
	\caption{ \label{}}
\end{figure}

\begin{figure}[p]
\centerline{\includegraphics[width=\textwidth]{ukbb_chains}}
\caption{ \label{}}
\end{figure}

\begin{figure}[p]
\centerline{\includegraphics[width=\textwidth]{ukbb_runtimes}}
\caption{ \label{}}
\end{figure}

%%%%%%%%%%%%%%%%%%%%%%%%%%%%%%%%%%%%%%%%%%%%%%%%%%%%%%%%%%%%%%%%%%%%%%%%%%%%%%%%
%%%%%%%%%%%%%%%%%%%%%%%%%%%%%%%%%%%%%%%%%%%%%%%%%%%%%%%%%%%%%%%%%%%%%%%%%%%%%%%%

\FloatBarrier

\begin{figure}[p]
	\centerline{\includegraphics[width=\textwidth]{ukbb_local_h2}}
	\caption{ \label{}}
\end{figure}

\begin{figure}[p]
	\centerline{\includegraphics[width=\textwidth]{protein_local_h2}}
	\caption{ \label{}}
\end{figure}

\FloatBarrier

%%%%%%%%%%%%%%%%%%%%%%%%%%%%%%%%%%%%%%%%%%%%%%%%%%%%%%%%%%%%%%%%%%%%%%%%%%%%%%%%

%%%%%%%%%%%%%%%%%%%%%%%%%%%%%%%%%%%%%%%%%%%%%%%%%%%%%%%%%%%%%%%%%%%%%%%%%%%%%%%%

\FloatBarrier
\clearpage

\bibliographystyle{natbib}
\bibliography{refs}

\end{document}
