%% LyX 1.3 created this file.  For more info, see http://www.lyx.org/.
%% Do not edit unless you really know what you are doing.
\documentclass[english, 12pt]{article}
\usepackage{times}
%\usepackage{algorithm2e}
\usepackage{xurl}
\usepackage{bbm}
\usepackage[T1]{fontenc}
\usepackage[latin1]{inputenc}
\usepackage{geometry}
\geometry{verbose,letterpaper,tmargin=2cm,bmargin=2cm,lmargin=2cm,rmargin=2cm}
\usepackage{rotating}
\usepackage{color}
\usepackage{graphicx}
\usepackage{amsmath, amsthm, amssymb}
\usepackage{setspace}
\usepackage{lineno}
\usepackage{hyperref}
\usepackage{bbm}
\usepackage{makecell}
%\usepackage[font=small]{caption}

\renewcommand{\arraystretch}{1.3}

\usepackage{xr}
\externaldocument{paper-infer-supp}

%\linenumbers
%\doublespacing
\onehalfspacing
\usepackage[authoryear]{natbib}
%\usepackage[numbers,super]{natbib} %\bibpunct{(}{)}{;}{author-year}{}{,}

%Pour les rajouts
\usepackage{color}
\definecolor{trustcolor}{rgb}{0,0,1}

\usepackage{dsfont}
\usepackage[warn]{textcomp}
\usepackage{adjustbox}
\usepackage{multirow}
\usepackage{graphicx}
\graphicspath{{../figures/}}
\DeclareMathOperator*{\argmin}{\arg\!\min}
\usepackage{algorithm}
\usepackage{algpseudocode}

\let\tabbeg\tabular
\let\tabend\endtabular
\renewenvironment{tabular}{\begin{adjustbox}{max width=\textwidth}\tabbeg}{\tabend\end{adjustbox}}

\makeatletter

%%%%%%%%%%%%%%%%%%%%%%%%%%%%%% LyX specific LaTeX commands.
%% Bold symbol macro for standard LaTeX users
%\newcommand{\boldsymbol}[1]{\mbox{\boldmath $#1$}}

%% Because html converters don't know tabularnewline
\providecommand{\tabularnewline}{\\}

\usepackage{babel}
\makeatother


\begin{document}


\title{Inferring disease architecture and predictive ability\\with LDpred2-auto}
\author{Florian Priv\'e,$^{\text{1,}*}$  and Bjarni J. Vilhj\'almsson$^{\text{1,2}}$}

\date{~ }
\maketitle

\noindent$^{\text{\sf 1}}$National Centre for Register-Based Research, Aarhus University, Aarhus, 8210, Denmark. \\
\noindent$^{\text{\sf 2}}$Bioinformatics Research Centre, Aarhus University, Aarhus, 8000, Denmark. \\
\noindent$^\ast$To whom correspondence should be addressed.\\

\noindent Contact: \url{florian.prive.21@gmail.com}

%\vspace*{6em}
\clearpage

\begin{abstract}

\end{abstract}

%%%%%%%%%%%%%%%%%%%%%%%%%%%%%%%%%%%%%%%%%%%%%%%%%%%%%%%%%%%%%%%%%%%%%%%%%%%%%%%%

\clearpage

%%%%%%%%%%%%%%%%%%%%%%%%%%%%%%%%%%%%%%%%%%%%%%%%%%%%%%%%%%%%%%%%%%%%%%%%%%%%%%%%

\section{Introduction}

Most traits and diseases are heritable, if not all. 
What differs is the proportion of phenotypic variance that can be attributable to genetics, i.e.\ the heritability of these phenotypes.
Some phenotypes, such as height and schizophrenia are highly heritable [TODO: ADD CITATIONS]. 
These two phenotypes are also highly polygenic, i.e.\ mutations from many genetic variants influences them [TODO: ADD CITATIONS].
Knowing how much a phenotype is heritable and polygenic can teach a lot about it.
Therefore, many methods have been developed to estimate the (SNP) heritability $h^2$ and polygenicity $p$.
These include GCTA ($h^2$), BOLT-REML ($h^2$ and $p$), LD Score regression ($h^2$), HESS (local $h^2$), LDAK-SumHer ($h^2$), S-LD4M ($p$), GRM-MAF-LD ($\alpha$ that can inform about negative selection), SBayesS ($h^2$, $p$, and a third parameter $S$, similar to $\alpha$), BEAVR (local $p$) \cite[]{yang2011gcta,loh2015contrasting,bulik2015ld,shi2016contrasting,speed2019sumher,oconnor2019extreme,schoech2019quantification,zeng2021widespread,johnson2021estimation}.

These two parameters are useful e.g.\ to determine how well we can predict a phenotype from using genetic variants alone [TODO: ADD CITATIONS]. 
Such genetic predictors are called polygenic scores (PGS), and are getting closer to be included as part of existing clinical risk models for diseases [TODO: ADD CITATIONS].
LDpred2 is a very competitive polygenic score method that can directly builds PGS using summary statistics results from genome-wide associations studies (GWAS), making it highly applicable \cite[]{prive2020ldpred2,pain2021evaluation,kulm2021systematic}.
LDpred2 uses the (SNP) heritability and polygenicity as parameters of its model. In LDpred2-auto, it can directly estimate these parameters from the data, making it applicable even when no validation data is available for tuning these two hyper-parameters of the model \cite[]{prive2020ldpred2}.

Here we extend LDpred2-auto to make it a highly reliable method for estimating (local) $h^2$, (per-variant) $p$, and $\alpha$.
We show that we can also reliably estimate the predictive ability of PGS derived from LDpred2-auto.

[TO FINISH]

[ALSO TALK ABOUT NEW SET OF VARIANTS]

%%%%%%%%%%%%%%%%%%%%%%%%%%%%%%%%%%%%%%%%%%%%%%%%%%%%%%%%%%%%%%%%%%%%%%%%%%%%%%%%


\section{Results}


\begin{figure}[!h]
	%\centerline{\includegraphics[width=0.95\textwidth]{res-FIN}}
	\caption{}
	\label{fig:finngen}
\end{figure}



%%%%%%%%%%%%%%%%%%%%%%%%%%%%%%%%%%%%%%%%%%%%%%%%%%%%%%%%%%%%%%%%%%%%%%%%%%%%%%%%

\section{Discussion}


[KEEP RG FOR DISCU + CITE EXISTING METHODS]

[BETTER SET OF VARIANTS, BUT STILL NOT PERFECT E.G. FOR PROTEIN LEVELS]

[ALSO TALK ABOUT IMPUTATION]

%%%%%%%%%%%%%%%%%%%%%%%%%%%%%%%%%%%%%%%%%%%%%%%%%%%%%%%%%%%%%%%%%%%%%%%%%%%%%%%%

\section{Materials and Methods}

\subsection{New model and inference with LDpred2-auto}

LDpred2 originally assumed the following model for effect sizes,
\begin{equation}\label{eq:prev_model}
\beta_j = S_j \gamma_j \sim \left\{
\begin{array}{ll}
\mathcal N\left(0,~\dfrac{h^2}{M p}\right) & \mbox{with probability $p$,} \\
0 & \mbox{otherwise,}
\end{array}
\right.
\end{equation}
where $p$ is the proportion of causal variants, $M$ the number of variants, $h^2$ the (SNP) heritability, $\gamma$ the effect sizes on the allele scale, $S$ the standard deviations of the genotypes, and $\beta$ the effects of the scaled genotypes \cite[]{prive2020ldpred2}.
In LDpred2-auto, $p$ and $h^2$ are directly estimated within the Gibbs sampler, as opposed to testing several values of $p$ and $h^2$ from a grid of hyper-parameters. This makes LDpred2-auto a method free of hyper-parameters which can therefore be applied directly to data without the need of a validation dataset to choose best-performing hyper-parameters \cite[]{prive2020ldpred2}.
Previously, $p$ was sampled from $\text{Beta}(1 + M_c, 1 + M - M_c)$, where $M_c = \sum_j(\beta_j \neq 0)$.

Here we introduce a few changes to LDpred2-auto, which makes it better at inferring these important parameters.
First, we extend LDpred2-auto with a third parameter $\alpha$ that controls the relationship between minor allele frequencies (or equivalently, standard deviations) of genotypes and expected effect sizes; the model becomes
\begin{equation}\label{eq:new_model}
\beta_j = S_j \gamma_j \sim \left\{
\begin{array}{ll}
\mathcal N \big( 0,~\sigma_\beta^2 \cdot (S_j^2)^{(\alpha + 1)} \big) & \mbox{with probability $p$,} \\
0 & \mbox{otherwise.}
\end{array}
\right.
\end{equation}
Therefore, it was earlier assumed that $\alpha = -1$ and $\sigma_\beta^2 = h^2 / (M p)$ in equation \eqref{eq:prev_model}. 
This new model in equation \eqref{eq:new_model} is similar to the model assumed by SBayesS, where $\alpha$ is called $S$ \cite{zeng2021widespread}. 
In SBayesS, they estimate $\alpha$ and $\sigma_\beta^2$ by maximizing the likelihood of the normal distribution (over the causal variants from the Gibbs sampler).
In the new LDpred2-auto, we first sample causal variants with replacement (bootstrap) before finding the maximum likelihood estimators, such that we add some proper sampling to these two parameters. 
This maximum likelihood estimation is implemented using R package roptim \cite[]{pan2020roptim}, and we bound the estimate of $\alpha$ to be within -1.5 and 0.5 (the default, but can be modified), and the estimate of $\sigma_\beta^2$ to be between 0.5 and 2 times the previous estimate.
We now sample $p$ from $\text{Beta}(1 + M_c / \bar{l^2}, 1 + (M - M_c) / \bar{l^2})$, where $\bar{l^2}$ is the average LD score, to add more sampling in order to account for the reduced effective number of correlated variants.
As for $h^2$, we still estimate it by $h^2 = \boldsymbol{\beta}^T \boldsymbol{R} \boldsymbol{\beta}$, where $\boldsymbol{R}$ is the correlation matrix between variants. We constrain this estimate to be at least 0.001 to prevent the Gibbs sampler from being trapped in very small heritability estimates.



\subsection{Extending the set of HapMap3 variants}


%%%%%%%%%%%%%%%%%%%%%%%%%%%%%%%%%%%%%%%%%%%%%%%%%%%%%%%%%%%%%%%%%%%%%%%%%%%%%%%%


%\clearpage
%\vspace*{3em}

\section*{Acknowledgements}

Authors also thank GenomeDK and Aarhus University for providing computational resources and support that contributed to these research results.
This research has been conducted using the UK Biobank Resource under Application Number 58024.

\section*{Funding}

F.P.\ and B.J.V.\ are supported by a Lundbeck Foundation Fellowship (R335-2019-2339 to B.J.V.).

\section*{Declaration of Interests}

B.J.V.\ is on Allelica's international advisory board.
The other authors have no competing interests to declare.

\section*{Code and data availability}

The UK Biobank data is available through a procedure described at \url{https://www.ukbiobank.ac.uk/using-the-resource/}. 
All code used for this paper is available at \url{https://github.com/privefl/paper-infer/tree/master/code}.
We have extensively used R packages bigstatsr and bigsnpr \cite[]{prive2017efficient} for analyzing large genetic data, packages from the future framework \cite[]{bengtsson2020unifying} for easy scheduling and parallelization of analyses on the HPC cluster, and packages from the tidyverse suite \cite[]{wickham2019welcome} for shaping and visualizing results.
The latest version of R package bigsnpr can be installed from GitHub, and a recent enough version can be installed from CRAN [TODO: VERIFY].

%%%%%%%%%%%%%%%%%%%%%%%%%%%%%%%%%%%%%%%%%%%%%%%%%%%%%%%%%%%%%%%%%%%%%%%%%%%%%%%%

\clearpage
%\vspace*{3em}

\bibliographystyle{natbib}
\bibliography{refs}

%%%%%%%%%%%%%%%%%%%%%%%%%%%%%%%%%%%%%%%%%%%%%%%%%%%%%%%%%%%%%%%%%%%%%%%%%%%%%%%%


\end{document}
